% Reduce the margins for this page
\newgeometry{left=2cm}

\section*{Memoriu activitate științifică}

\subsection*{Școli de vară}

Am participat la următoarele școli de vară:

\begin{itemize}
    \item \href{https://sites.google.com/site/marinliviu/french-romanian-summer-school-on-applied-mathematics/8th-french-romanian-summer-school-on-applied-mathematics-5-13-july-2023?authuser=0}{\textbf{French-Romanian Summer School on Applied Mathematics}}, 5 -- 13 iulie 2023, Sinaia, România. \textbf{Cursuri:} Regularity of Water Waves, Markov Chain Monte Carlo, Microstructures, Random Matrix Theory.

    \item \href{https://www.smi-math.unipr.it/past-courses/perugia/summer-school-smi-2022/100/}{\textbf{Scuola Matematica Interuniversitaria Summer School}}, 18 iulie -- 12 august 2022, Perugia, Italia. \textbf{Cursuri:} Algebraic Geometry, Functional Analysis.
\end{itemize}

\subsection*{Seminare științifice și conferințe}

Am participat la următoarele seminare științifice sau conferințe:

\begin{itemize}
    \item \href{https://sites.google.com/view/ai-unibuc/home}{\textbf{AI Reading Group @ Unibuc}},  organizat de Facultatea de Matematică și Informatică (FMI) a Universității din București. \textbf{Topici:} machine learning, computer vision, natural language processing, reinforcement learning.
    
    \item \href{https://sites.google.com/view/master-probabilitati-fmi/seminarul-stiintific-probabilitati-si-teme-conexe}{\textbf{Probability Seminar}}, organizat de FMI și de Institutul de Matematică al Academiei Române (IMAR). \textbf{Topici:} geometrie tropicală, modele de limbă, analiza componentelor independente, regularizare entropică.

    \item \textbf{Conferințele lunare} ale FMI. \textbf{Topici:} topologie, geometrie diferențială, geometrie complexă, analiză armonică, curbe eliptice, metode Monte Carlo, matematica găurilor negre, teoria spectrală a grafurilor.

    \item Conferința \href{https://nlp.unibuc.ro/events/raai2019.html}{\textbf{Recent Advances in Artifical Intelligence}}, ediția 2019.
\end{itemize}

\subsection*{Concursuri}

\begin{itemize}
    \item Premiul III la \href{https://fmi.unibuc.ro/sesiune-de-comunicari-stiintifice-studentesti/}{Sesiunea de Comunicări Științifice Studențești} din cadrul FMI-UB, ediția 2023.

    \item Premiul Innovation oferit de Autodesk în cadrul hackathon-ului \href{https://smarthack.asmi.ro/}{SMARTHACK}, ediția 2022.

    \item Premiul II în cadrul concursului LSEG Romania - Quant Challenge, ediția 2022, organizat de London Stock Exchange Group pentru studenții la master.

    \item Premiul III la \href{https://fmi.unibuc.ro/comunicari-stiintifice-studentesti-2021/}{Sesiunea de Comunicări Științifice Studențești} din cadrul FMI-UB, ediția 2021.
\end{itemize}

\subsection*{Educație}

\begin{itemize}
    \item \textbf{Master} în \textbf{Matematică}, Facultatea de Matematică și Informatică, Universitatea din București, promoția 2023.

    \item \textbf{Licență} în \textbf{Informatică}, Facultatea de Matematică și Informatică, Universitatea din București, promoția 2021.

    \item Diplomă de bacalaureat obținută în urma absolvirii studiilor liceale la Colegiul Național Mihai Viteazul, București --- Specializarea matematică-informatică, intensiv informatică, promoția 2018.
\end{itemize}

\subsection*{Predare}

Am ținut ore de seminar și de laborator pentru materiile de \emph{Procesarea Semnalelor}, \emph{Algoritmi Avansați} și \emph{Programare Orientată pe Obiecte} din cadrul FMI-UB.
