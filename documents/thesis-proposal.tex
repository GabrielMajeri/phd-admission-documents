% Reduce the margins for this page
\newgeometry{left=2.5cm,bottom=1cm}

\section*{Propunere temă de doctorat}

\textbf{Titlul complet al temei:} Studiul dinamicii sistemelor complexe folosind metode computaționale avansate \\[0.5em]
\textbf{Direcția:} Fizică teoretică și computațională \\[0.5em]
\textbf{Coordonator:} Prof.~dr.~Virgil Băran \\[0.5em]
\textbf{Candidat:} Gabriel Majeri

\subsection*{Motivarea temei}

Cercetarea științifică modernă se bazează tot mai mult pe tehnici computaționale. Modelele teoretice implementate pe calculator pot valida sau ghida eforturile experimentale. Astfel de modele sunt esențiale pentru înțelegerea sistemelor complexe, care deseori nu admit soluții analitice.

Pe lângă metodele consacrate de calcul numeric, în ultimii ani au crescut în popularitate metodele bazate pe învățare automată sau pe calcul cuantic. Utilizarea cu succes a unor astfel de instrumente necesită cunoștințe interdisciplinare și o bună colaborare între teoreticieni și tehnicieni.

Îmi propun să folosesc astfel de metode computaționale avansate pentru a studia și eventual rezolva probleme de interes atât pentru cercetarea fundamentală (fizica particulelor, cosmologie etc.), cât și pentru cea aplicată (biofizică, fizica materialelor etc). Scopul meu final este să contribui la progresul științific și la rezolvarea unor probleme de interes general, cum ar fi schimbările climatice sau bolile incurabile.

\subsection*{Etapele propuse}

\begin{itemize}
    \item Familiarizarea cu grupurile de cercetare și cu infrastructura disponibilă în cadrul Facultății de Fizică și în institutele asociate.

    \item Aprofundarea cunoștințelor de biologie, chimie și fizică, urmărind cu atenție aspectele computaționale implicate în aceste domenii.

    \item Parcurgerea unor lucrări științifice din jurnale de top, propuse de către coordonatorul de doctorat sau alți profesori, în vederea familiarizării cu stadiul progresului actual din domeniile respective.

    \item Studierea, înțelegerea și extinderea unor pachete computaționale folosite în practică, precum și optimizarea lor pentru execuția pe infrastructura computațională a Universității.
\end{itemize}

\clearpage

\section*{Referințe bibliografice}

\nocite{*}

\printbibliography[heading=none]

\clearpage
